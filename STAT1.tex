\documentclass[]{book}
\usepackage{lmodern}
\usepackage{amssymb,amsmath}
\usepackage{ifxetex,ifluatex}
\usepackage{fixltx2e} % provides \textsubscript
\ifnum 0\ifxetex 1\fi\ifluatex 1\fi=0 % if pdftex
  \usepackage[T1]{fontenc}
  \usepackage[utf8]{inputenc}
\else % if luatex or xelatex
  \ifxetex
    \usepackage{mathspec}
  \else
    \usepackage{fontspec}
  \fi
  \defaultfontfeatures{Ligatures=TeX,Scale=MatchLowercase}
\fi
% use upquote if available, for straight quotes in verbatim environments
\IfFileExists{upquote.sty}{\usepackage{upquote}}{}
% use microtype if available
\IfFileExists{microtype.sty}{%
\usepackage{microtype}
\UseMicrotypeSet[protrusion]{basicmath} % disable protrusion for tt fonts
}{}
\usepackage[margin=1in]{geometry}
\usepackage{hyperref}
\hypersetup{unicode=true,
            pdftitle={Stat1: statistische modellen},
            pdfauthor={Mark Smits},
            pdfborder={0 0 0},
            breaklinks=true}
\urlstyle{same}  % don't use monospace font for urls
\usepackage{color}
\usepackage{fancyvrb}
\newcommand{\VerbBar}{|}
\newcommand{\VERB}{\Verb[commandchars=\\\{\}]}
\DefineVerbatimEnvironment{Highlighting}{Verbatim}{commandchars=\\\{\}}
% Add ',fontsize=\small' for more characters per line
\usepackage{framed}
\definecolor{shadecolor}{RGB}{248,248,248}
\newenvironment{Shaded}{\begin{snugshade}}{\end{snugshade}}
\newcommand{\KeywordTok}[1]{\textcolor[rgb]{0.13,0.29,0.53}{\textbf{{#1}}}}
\newcommand{\DataTypeTok}[1]{\textcolor[rgb]{0.13,0.29,0.53}{{#1}}}
\newcommand{\DecValTok}[1]{\textcolor[rgb]{0.00,0.00,0.81}{{#1}}}
\newcommand{\BaseNTok}[1]{\textcolor[rgb]{0.00,0.00,0.81}{{#1}}}
\newcommand{\FloatTok}[1]{\textcolor[rgb]{0.00,0.00,0.81}{{#1}}}
\newcommand{\ConstantTok}[1]{\textcolor[rgb]{0.00,0.00,0.00}{{#1}}}
\newcommand{\CharTok}[1]{\textcolor[rgb]{0.31,0.60,0.02}{{#1}}}
\newcommand{\SpecialCharTok}[1]{\textcolor[rgb]{0.00,0.00,0.00}{{#1}}}
\newcommand{\StringTok}[1]{\textcolor[rgb]{0.31,0.60,0.02}{{#1}}}
\newcommand{\VerbatimStringTok}[1]{\textcolor[rgb]{0.31,0.60,0.02}{{#1}}}
\newcommand{\SpecialStringTok}[1]{\textcolor[rgb]{0.31,0.60,0.02}{{#1}}}
\newcommand{\ImportTok}[1]{{#1}}
\newcommand{\CommentTok}[1]{\textcolor[rgb]{0.56,0.35,0.01}{\textit{{#1}}}}
\newcommand{\DocumentationTok}[1]{\textcolor[rgb]{0.56,0.35,0.01}{\textbf{\textit{{#1}}}}}
\newcommand{\AnnotationTok}[1]{\textcolor[rgb]{0.56,0.35,0.01}{\textbf{\textit{{#1}}}}}
\newcommand{\CommentVarTok}[1]{\textcolor[rgb]{0.56,0.35,0.01}{\textbf{\textit{{#1}}}}}
\newcommand{\OtherTok}[1]{\textcolor[rgb]{0.56,0.35,0.01}{{#1}}}
\newcommand{\FunctionTok}[1]{\textcolor[rgb]{0.00,0.00,0.00}{{#1}}}
\newcommand{\VariableTok}[1]{\textcolor[rgb]{0.00,0.00,0.00}{{#1}}}
\newcommand{\ControlFlowTok}[1]{\textcolor[rgb]{0.13,0.29,0.53}{\textbf{{#1}}}}
\newcommand{\OperatorTok}[1]{\textcolor[rgb]{0.81,0.36,0.00}{\textbf{{#1}}}}
\newcommand{\BuiltInTok}[1]{{#1}}
\newcommand{\ExtensionTok}[1]{{#1}}
\newcommand{\PreprocessorTok}[1]{\textcolor[rgb]{0.56,0.35,0.01}{\textit{{#1}}}}
\newcommand{\AttributeTok}[1]{\textcolor[rgb]{0.77,0.63,0.00}{{#1}}}
\newcommand{\RegionMarkerTok}[1]{{#1}}
\newcommand{\InformationTok}[1]{\textcolor[rgb]{0.56,0.35,0.01}{\textbf{\textit{{#1}}}}}
\newcommand{\WarningTok}[1]{\textcolor[rgb]{0.56,0.35,0.01}{\textbf{\textit{{#1}}}}}
\newcommand{\AlertTok}[1]{\textcolor[rgb]{0.94,0.16,0.16}{{#1}}}
\newcommand{\ErrorTok}[1]{\textcolor[rgb]{0.64,0.00,0.00}{\textbf{{#1}}}}
\newcommand{\NormalTok}[1]{{#1}}
\usepackage{longtable,booktabs}
\usepackage{graphicx,grffile}
\makeatletter
\def\maxwidth{\ifdim\Gin@nat@width>\linewidth\linewidth\else\Gin@nat@width\fi}
\def\maxheight{\ifdim\Gin@nat@height>\textheight\textheight\else\Gin@nat@height\fi}
\makeatother
% Scale images if necessary, so that they will not overflow the page
% margins by default, and it is still possible to overwrite the defaults
% using explicit options in \includegraphics[width, height, ...]{}
\setkeys{Gin}{width=\maxwidth,height=\maxheight,keepaspectratio}
\IfFileExists{parskip.sty}{%
\usepackage{parskip}
}{% else
\setlength{\parindent}{0pt}
\setlength{\parskip}{6pt plus 2pt minus 1pt}
}
\setlength{\emergencystretch}{3em}  % prevent overfull lines
\providecommand{\tightlist}{%
  \setlength{\itemsep}{0pt}\setlength{\parskip}{0pt}}
\setcounter{secnumdepth}{5}
% Redefines (sub)paragraphs to behave more like sections
\ifx\paragraph\undefined\else
\let\oldparagraph\paragraph
\renewcommand{\paragraph}[1]{\oldparagraph{#1}\mbox{}}
\fi
\ifx\subparagraph\undefined\else
\let\oldsubparagraph\subparagraph
\renewcommand{\subparagraph}[1]{\oldsubparagraph{#1}\mbox{}}
\fi

%%% Use protect on footnotes to avoid problems with footnotes in titles
\let\rmarkdownfootnote\footnote%
\def\footnote{\protect\rmarkdownfootnote}

%%% Change title format to be more compact
\usepackage{titling}

% Create subtitle command for use in maketitle
\newcommand{\subtitle}[1]{
  \posttitle{
    \begin{center}\large#1\end{center}
    }
}

\setlength{\droptitle}{-2em}

  \title{Stat1: statistische modellen}
    \pretitle{\vspace{\droptitle}\centering\huge}
  \posttitle{\par}
    \author{Mark Smits}
    \preauthor{\centering\large\emph}
  \postauthor{\par}
      \predate{\centering\large\emph}
  \postdate{\par}
    \date{2019-07-10}

\usepackage{booktabs}

\usepackage{amsthm}
\newtheorem{theorem}{Theorem}[chapter]
\newtheorem{lemma}{Lemma}[chapter]
\newtheorem{corollary}{Corollary}[chapter]
\newtheorem{proposition}{Proposition}[chapter]
\newtheorem{conjecture}{Conjecture}[chapter]
\theoremstyle{definition}
\newtheorem{definition}{Definition}[chapter]
\theoremstyle{definition}
\newtheorem{example}{Example}[chapter]
\theoremstyle{definition}
\newtheorem{exercise}{Opdracht}[chapter]
\theoremstyle{remark}
\newtheorem*{remark}{Remark}
\newtheorem*{solution}{Solution}
\let\BeginKnitrBlock\begin \let\EndKnitrBlock\end
\begin{document}
\maketitle

{
\setcounter{tocdepth}{1}
\tableofcontents
}
\chapter*{Voorwoord}\label{voorwoord}
\addcontentsline{toc}{chapter}{Voorwoord}

Afgelopen blok hebben jullie al aardig wat data verzameld met practica,
en de meeste ook al met hun bioxperience-project. In dit blok leren
jullie om op basis van data betrouwbare uitspraken te doen. Zoals je
gemerkt zult hebben in de practica, krijgt iedere groep een iets andere
uitkomst uit precies hetzelfde experiment. Wat is nu de waarheid?
Statistiek is ervoor om de betrouwbaarheid van de uitkomsten te
berekenen. Op die manier weet je wat je wel en niet kan beweren op basis
van je onderzoek.

\section*{Opzet}\label{opzet}
\addcontentsline{toc}{section}{Opzet}

Blok 4 is opgebouwd uit zes werkcolleges van ieder anderhalf uur en een
hoorcollege aan het eind. Ieder werkcollege begint met een korte
inleiding over het onderwerp, gevolgd door een opdracht die je in
groepjes uitvoert en mogelijk ook individuele opdrachten. Houd van de
opdrachten een portfolio bij.

We gaan dit blok het boek \emph{The Analysis of Biological Data}
gebruiken. Jullie worden geacht \textbf{vooraf} de aangegeven
hoofdstukken gelezen te hebben!

\chapter{Werkcollege 1: Statistische vragen bij
grafieken}\label{werkcollege-1-statistische-vragen-bij-grafieken}

\BeginKnitrBlock{ABD}
Lees Chapter 1 (\emph{Statistics and samples}):

Lees Chapter 2 (\emph{Displaying data}):
\EndKnitrBlock{ABD}

\section{Soorten grafieken}\label{soorten-grafieken}

Een grafiek is dé manier om data te presenteren. Het laat in één
oogopslag zien wat voor patronen er in de data zitten. Verschillende
soorten data vragen om verschillende soorten grafieken.

Een aantal mogelijke soorten grafieken:

\begin{itemize}
\tightlist
\item
  Boxplot
\item
  Spreidingsdiagram
\item
  Staafdiagram
\item
  Taartdiagram/mozaiekplot
\end{itemize}

\BeginKnitrBlock{exercise}
\protect\hypertarget{exr:figuren}{}{\label{exr:figuren} }Soorten figuren

\begin{itemize}
\tightlist
\item
  Zoek in de literatuur, die je verzameld hebt voor bioxperience, naar
  voorbeelden voor ieder van bovenstaande typen figuren, en kopieer ze
  naar je portfolio. Komt niet iedere soort grafiek voor, dan vindt je
  op blackboard een serie artikelen waar ze wel in voorkomen.

  \EndKnitrBlock{exercise}
\end{itemize}

Data op de juiste manier presenteren is de eerste stap in het analyseren
van je data.

\section{Soorten data}\label{soorten-data}

\BeginKnitrBlock{exercise}
\protect\hypertarget{exr:dataniveaus}{}{\label{exr:dataniveaus} }Soorten
data

Data kan verdeeld worden in drie niveau's.

\begin{itemize}
\tightlist
\item
  Zoek in \emph{Analysis of Biological Data} op welke niveau's dat zijn,
  noteer ze in je portfolio.
\item
  Benoem de niveaus van de data in de figuren die je in de vorige
  opdracht hebt verzameld.

  \EndKnitrBlock{exercise}
\end{itemize}

\section{Statistische vragen}\label{statistische-vragen}

Achter ieder onderzoek, en dus ook achter iedere dataset, zitten een of
meerdere onderzoeksvragen. De kunst is om een figuur zo op te zetten dat
de vraag daarmee beantwoord kan worden. Een onderzoeksvraag wordt
daarmee automatisch ook een statistische vraag omdat je met data uit een
steekproef werkt. Daarmee heb je te maken met onzekerheid.

\BeginKnitrBlock{exercise}
\protect\hypertarget{exr:onderzoeksvragen}{}{\label{exr:onderzoeksvragen}
}Onderzoeksvragen

\begin{itemize}
\tightlist
\item
  Zoek uit welke onderzoeksvragen horen bij de figuren die je hebt
  uitgezocht in de vorige opdrachten.
\item
  Zoek uit (tip: check materiaal en methode) welke statistische toets
  gebruikt is om de onderzoeksvraag te beantwoorden.
\item
  Wat is de conclusie van de statistische toets?
\item
  Noteer alle antwoorden in je portfolio.

  \EndKnitrBlock{exercise}
\end{itemize}

\BeginKnitrBlock{exercise}
\protect\hypertarget{exr:oef_h1_2}{}{(\#exr:oef\_h1\_2) }Maak de
volgende opgaven uit het boek:

\begin{itemize}
\tightlist
\item
  1.2
\item
  2.6
\item
  2.8
\item
  2.11
\item
  2.15

  \EndKnitrBlock{exercise}
\end{itemize}

\chapter{Werkcollege 2: Statistische
toetsen}\label{werkcollege-2-statistische-toetsen}

\BeginKnitrBlock{ABD}
Lees Chapter 6 (\emph{Hypothesis testing})
\EndKnitrBlock{ABD}

Belangrijkste leerdoelen:

\begin{itemize}
\tightlist
\item
  Hypotheses kunnen opstellen
\item
  Kunnen omschrijven wat de p-waarde is die uit een statistische toets
  rolt
\item
  Begrijpen wat fouten van eerste (type I) en tweede (type II) soort
  zijn.
\end{itemize}

\section{Hypotheses}\label{hypotheses}

Statistische toetsen zijn ontwikkeld om helder te krijgen hoe
betrouwbaar een bepaalde uitspraak is. In blok 2 hebben jullie geleerd
om vanuit een steekproef het betrouwbaarheidsinterval te berekenen van
het gemiddelde. Je geeft daarmee aan dat het gemiddelde, met een
bepaalde waarschijnlijkheid (meestal 95\%) tussen een minimale en
maximale waarde zit.

Met een statistische toets, doe je precies het omgekeerde. In plaats van
het bepalen van het betrouwbaarheidsinterval, bepaal je de
waarschijnlijkheid dat iets gelijk is aan een preciese waarde (je
veronderstelling). De vergelijking die je maakt wordt de nul-hypothese
genoemd. Wanneer het helemaal niet waarschijnlijk is dat je
nul-hypothese waar is, verwerp je deze. Je neemt dan de alternatieve
hypothese aan. De alternatieve hypothese is nooit precies (bijv. het
gemiddelde is niet gelijk aan 10).

Meestal is de nul-hypothese niet jouw onderzoekshypothese! Meestal
verwacht je een verschil tussen bijv. twee behandelingen, maar je
nul-hypothese stelt dat er geen verschil is.

\BeginKnitrBlock{exercise}
\protect\hypertarget{exr:hypotheses}{}{\label{exr:hypotheses} }Hypotheses
opstellen

\begin{itemize}
\item
  Maak de volgende opgaven:

  \begin{itemize}
  \tightlist
  \item
    Opgave 15 (blz. 170)
  \item
    Opgave 16 (blz. 170)
  \item
    Ogave 17 (blz. 171)

    \EndKnitrBlock{exercise}
  \end{itemize}
\end{itemize}

\section{Type I en type II fouten}\label{type-i-en-type-ii-fouten}

Je hoopt natuurlijk altijd de juiste conclusie te trekken uit je data.
Helaas is dat niet altijd het geval. Gelukkig kunnen we met statistiek
wel aangeven hoe betrouwbaar je conclusie is.

De meeste kritieke fout is het ten onrechte verwerpen van de
nul-hypothese: je toont een verschil aan die er niet is. De kans op deze
fout wordt (bijna) standaard op 5\% gezet. De p-waarde geeft de
waarschijnlijkheid dat de nul-hypothese \textbf{waar} is. Komt de
p-waarde onder de 0,05 (=5\%), dan verwerpen we de nul-hypothese.

Een irritante fout is als er echt wel een verschil is (dus de
nul-hypothese is \textbf{niet} waar), maar je kan het niet aantonen. Dit
wordt een fout van de tweede soort genoemd. Hoe strenger je bent voor de
kans op een fout van de eerste soort, hoe groter de kans op een fout van
de tweede soort (denk daar maar even over na!).

\BeginKnitrBlock{exercise}
\protect\hypertarget{exr:typefouten}{}{\label{exr:typefouten} }Type I en
type II fouten

\begin{itemize}
\item
  Beschrijf in eigen woorden de betekenis van:

  \begin{itemize}
  \tightlist
  \item
    p-waarde
  \item
    fout van de eerste soort
  \item
    fout van de tweede soort

    \EndKnitrBlock{exercise}
  \end{itemize}
\end{itemize}

\section{Eenzijdig en tweezijdig
toetsen}\label{eenzijdig-en-tweezijdig-toetsen}

Als je wilt testen of er een verschil is tussen twee groepen, dan test
je tweezijdig. Het maakt niet uit welke van de twee groter is, als er
maar verschil is.

Soms ben je alleen geïnteresseerd of er een verschil in één richting is,
bijv. of biologen hoger scoren voor biologie dan voor wiskunde. Dan ben
je eenzijdig aan het toetsen.

\section{Simuleren}\label{simuleren}

De beste manier om begrip te krijgen van statistische toetsen, is om met
data te simuleren. R is hier ideaal voor. Met de functie
\texttt{sample()} kan je een random sample nemen uit een populatie:

\begin{Shaded}
\begin{Highlighting}[]
\NormalTok{steekproef1 <-}\StringTok{ }\KeywordTok{sample}\NormalTok{(}\KeywordTok{c}\NormalTok{(}\StringTok{"links"}\NormalTok{, }\StringTok{"rechts"}\NormalTok{), }\DecValTok{18}\NormalTok{, }\DataTypeTok{replace =} \OtherTok{TRUE}\NormalTok{)}
\end{Highlighting}
\end{Shaded}

De functie \texttt{sample()} heeft in bovenstaand voorbeeld 3
\textbf{argumenten}. Het eerste argument is een \textbf{vector} met de
mogelijke waarden in de populatie, het tweede argument geeft de grootte
van de steekproef en het derde argument geeft aan dat er een oneindige
populatie is met ``links''- en``rechts''-waarden. De functie neemt aan
dat ``links'' en ``rechts'' beiden evenveel voorkomen.

Met de table-functie kan je gemakkelijk checken hoeveel rechtshandige
padden er in de steekproef zitten:

\begin{Shaded}
\begin{Highlighting}[]
\KeywordTok{table}\NormalTok{(steekproef1)}
\end{Highlighting}
\end{Shaded}

\BeginKnitrBlock{exercise}
\protect\hypertarget{exr:steekproef}{}{\label{exr:steekproef} }Steekproef

\begin{itemize}
\tightlist
\item
  Gebruik bovenstaande code om een steekproef te nemen van 18 padden
\item
  Geef het aantal rechtshandige padden door aan de docent. Of via de
  volgende link

  \EndKnitrBlock{exercise}
\end{itemize}

Stel dat er wel een voorkeur voor rechtshandigheid is bij padden. Zeg
dat 80\% van de padden rechtshandig is.

Met een iets aangepaste code kunnen we weer een steekproef nemen, maar
nu van deze populatie:

\begin{Shaded}
\begin{Highlighting}[]
\NormalTok{steekproef2 <-}\StringTok{ }\KeywordTok{sample}\NormalTok{(}\KeywordTok{c}\NormalTok{(}\StringTok{"links"}\NormalTok{, }\StringTok{"rechts"}\NormalTok{), }\DecValTok{18}\NormalTok{, }\DataTypeTok{replace =} \OtherTok{TRUE}\NormalTok{, }\DataTypeTok{prob =} \KeywordTok{c}\NormalTok{(}\FloatTok{0.2}\NormalTok{, }\FloatTok{0.8}\NormalTok{))}
\end{Highlighting}
\end{Shaded}

\BeginKnitrBlock{exercise}
\protect\hypertarget{exr:steekproef2}{}{\label{exr:steekproef2} }Steekproef2

\begin{itemize}
\tightlist
\item
  Neem nu een steekproef van de populatie padden met 80\%
  rechtshandigheid.
\item
  Geef het aantal rechtshandige padden door aan de docent. Of via de
  link.

  \EndKnitrBlock{exercise}
\end{itemize}

Je ziet dat het moeilijk is om de juiste conclusie te trekken uit een
steekproef. Er zit altijd een mate van onzekerheid omheen.

De oplossing is om die onzekerheid goed te omschrijven. Met een
statistische toets neem je de kansverdeling onder de nulhypothese als
basis. In het geval van de padden gaan we onder de nulhypothese er van
uit dat er geen verschil is tussen kans op links- of rechtshandigheid
bij padden. Met R code kunnen we die kansverdeling berekenen en in een
histogram weergeven:

\begin{Shaded}
\begin{Highlighting}[]
\KeywordTok{library}\NormalTok{(tidyverse)}

\CommentTok{#10000 steekproeven nemen, resultaat is een matrix}
\NormalTok{samples <-}\StringTok{ }\KeywordTok{replicate}\NormalTok{(}\DecValTok{10000}\NormalTok{, }\KeywordTok{sample}\NormalTok{(}\KeywordTok{c}\NormalTok{(}\StringTok{"links"}\NormalTok{, }\StringTok{"rechts"}\NormalTok{), }\DecValTok{18}\NormalTok{, }\DataTypeTok{replace =} \OtherTok{TRUE}\NormalTok{))}
\CommentTok{#Matrix omzetten naar een dataframe}
\NormalTok{samples <-}\StringTok{ }\KeywordTok{data.frame}\NormalTok{(samples) %>%}\StringTok{ }\KeywordTok{gather}\NormalTok{(}\DataTypeTok{key =} \StringTok{"steekproef"}\NormalTok{, }\DataTypeTok{value =} \StringTok{"voorkeur"}\NormalTok{)}

\CommentTok{#Aantal rechtshandigen per steekproef}
\NormalTok{rechts <-}\StringTok{ }\NormalTok{samples %>%}\StringTok{ }
\StringTok{  }\KeywordTok{filter}\NormalTok{(voorkeur ==}\StringTok{ "rechts"}\NormalTok{) %>%}\StringTok{ }
\StringTok{  }\KeywordTok{group_by}\NormalTok{(steekproef) %>%}\StringTok{ }
\StringTok{  }\KeywordTok{count}\NormalTok{()}

\CommentTok{#Histogram maken van het gevonden aantal rechtshandige padden per steekproef}
\NormalTok{rechts %>%}\StringTok{ }
\StringTok{  }\KeywordTok{ggplot}\NormalTok{(}\KeywordTok{aes}\NormalTok{(n)) +}
\StringTok{  }\KeywordTok{geom_histogram}\NormalTok{(}\DataTypeTok{binwidth =} \DecValTok{1}\NormalTok{, }\DataTypeTok{center =} \DecValTok{0}\NormalTok{, }
    \DataTypeTok{fill =} \StringTok{"white"}\NormalTok{, }\DataTypeTok{colour =} \StringTok{"black"}\NormalTok{) +}
\StringTok{  }\KeywordTok{xlab}\NormalTok{(}\StringTok{"aantal rechtshandige padden"}\NormalTok{) +}
\StringTok{  }\KeywordTok{theme_classic}\NormalTok{()}
\end{Highlighting}
\end{Shaded}

Met de volgende functie kan je de kans berekenen (onder de nulhypothese)
op minstens het aantal rechtshandige padden, als in je eigen steekproef
(als voorbeeld minstens 14):

\begin{Shaded}
\begin{Highlighting}[]
\CommentTok{#Op basis van gesimuleerde kansverdeling}
\KeywordTok{sum}\NormalTok{(rechts$n>=}\DecValTok{14}\NormalTok{)/}\KeywordTok{length}\NormalTok{(rechts$n)}

\CommentTok{#Op basis van theoretische kansverdeling}
\KeywordTok{binom.test}\NormalTok{(}\DecValTok{14}\NormalTok{,}\DecValTok{18}\NormalTok{, }\DataTypeTok{alternative =} \StringTok{"greater"}\NormalTok{)}
\end{Highlighting}
\end{Shaded}

\BeginKnitrBlock{exercise}
\protect\hypertarget{exr:stattoets}{}{\label{exr:stattoets} }Statistische
toets

\begin{itemize}
\tightlist
\item
  Gebruik bovenstaande code om uit te rekenen wat de kans is op minstens
  zoveel rechtshandige padden als je gevonden hebt in je steekproef (doe
  dat voor beide steekproeven).
\item
  Stel de drempelwaarde van de kans op fout van de eerste soort op 0,05.
  Wat is dan je conclusie bij beide steekproeven. Geef dat door aan de
  docent. Of via de link.
\item
  Als iedereen het heeft uitgevoerd:

  \begin{itemize}
  \tightlist
  \item
    Hoeveel procent van de studenten verwerpt ten onrechte de
    nul-hypothese van de eerste steekproef (type I fout)?
  \item
    Hoeveel procent van de studenten verwerpt niet de nulhypothese (type
    II fout)?

    \EndKnitrBlock{exercise}
  \end{itemize}
\end{itemize}

\BeginKnitrBlock{exercise}
\protect\hypertarget{exr:oefeningen_H6}{}{(\#exr:oefeningen\_H6) } * 6.1
* 6.2 * 6.3 * 6.7 * 6.8 * 6.10 * 6.12
\EndKnitrBlock{exercise}

\chapter{Normale verdeling}\label{normale-verdeling}

\BeginKnitrBlock{ABD}
Lees Chapter 10 (\emph{Normal distribution}), behalve 10.2, 10.4 en
10.7.
\EndKnitrBlock{ABD}

Zoals jullie in blok 2 hebben geleerd, gedraagt veel variatie zich op
een karakteristieke manier die omschreven kan worden als de
\textbf{normale verdeling}.

Ter illustratie hieronder een histogram van de gemiddelde cijfers van
het HAVO-eindexamen biologie per school weergegeven. Je ziet dat de
verdeling min of meer een normale verdeling volgt (de zwarte lijn).

\includegraphics{{[}03{]}_normal_distribution_files/figure-latex/unnamed-chunk-2-1.pdf}

\BeginKnitrBlock{exercise}
\protect\hypertarget{exr:nverdeling}{}{\label{exr:nverdeling} }normale
verdeling

\begin{itemize}
\tightlist
\item
  Op basis van welke twee parameters is bovenstaande normale verdeling
  gebaseerd?
\end{itemize}
\EndKnitrBlock{exercise}

\section{R als alternatief voor een
tabel}\label{r-als-alternatief-voor-een-tabel}

In blok 2 heb je, met behulp van een tabel, de volgende vraag op kunnen
lossen:

\begin{quote}
Koeien van het ras Holstein-Friesian hebben gemiddeld een gewicht van
540 kg, met een standaarddeviatie van 5 kg. De vraag is: Hoe hoog is de
fractie koeien van het ras Holstein-Friesian met een gewicht tussen 542
en 545 kg?
\end{quote}

Hoe pakte je dat aan in blok 2? Het gaat om de volgende verdeling:
\(\underline{x} \sim N(\mu=540, \sigma=5)\). Fractie koeien tussen 542
en 545 kg: \(P(542<\underline{x}<545)\).

Volgende stap is omrekenen naar de standaard normale verdeling, je
berekent dan de z-waardes van 542 en 545 kg:
\(P \left( \frac{542-540}{5}<Z< \frac{545-540}{5} \right)\).

Vervolgens zoek je in de tabel de kans op
\(P \left( Z>\frac{2}{5} \right)\) en \(P \left( Z>\frac{5}{5} \right)\)
en trek deze van elkaar af: \(0.3446-0.1587 = 0.1859\)

In plaats van de tabel, kan je ook R-code gebruiken: \texttt{pnorm()}.
NB: deze functie berekent standaard
\(P \left( \underline{x}<q \right)\), dus de \emph{lower tail} als je
een normale verdeling bekijkt. Wil je het oppervlak aan de rechterkant
van q weten, voeg dan het argument \texttt{lower.tail\ =\ FALSE} toe.
Verder kan je ook aangeven wat het gemiddelde en de standaarddeviatie
is, je hoeft dus niet eerst de z-waarden uit te rekenen. Wil je nu weten
welke fractie koeien tussen 542 en 545 kg is, voer dan de volgende code
uit:

\begin{Shaded}
\begin{Highlighting}[]
\KeywordTok{pnorm}\NormalTok{(}\DecValTok{542}\NormalTok{, }\DataTypeTok{mean =} \DecValTok{540}\NormalTok{, }\DataTypeTok{sd =} \DecValTok{5}\NormalTok{, }\DataTypeTok{lower.tail =} \OtherTok{FALSE}\NormalTok{) -}\StringTok{ }
\StringTok{  }\KeywordTok{pnorm}\NormalTok{(}\DecValTok{545}\NormalTok{, }\DataTypeTok{mean =} \DecValTok{540}\NormalTok{, }\DataTypeTok{sd =} \DecValTok{5}\NormalTok{, }\DataTypeTok{lower.tail =} \OtherTok{FALSE}\NormalTok{)}
\end{Highlighting}
\end{Shaded}

Of korter:

\begin{Shaded}
\begin{Highlighting}[]
\KeywordTok{pnorm}\NormalTok{(}\DecValTok{542}\NormalTok{, }\DecValTok{540}\NormalTok{, }\DecValTok{5}\NormalTok{, }\DataTypeTok{lower.tail =} \OtherTok{FALSE}\NormalTok{) -}
\StringTok{  }\KeywordTok{pnorm}\NormalTok{(}\DecValTok{545}\NormalTok{, }\DecValTok{540}\NormalTok{, }\DecValTok{5}\NormalTok{, }\DataTypeTok{lower.tail =} \OtherTok{FALSE}\NormalTok{)}
\end{Highlighting}
\end{Shaded}

\BeginKnitrBlock{exercise}
\protect\hypertarget{exr:koe1}{}{\label{exr:koe1} }Koeien, deel 1

\begin{itemize}
\tightlist
\item
  Schets de normale verdeling van het gewicht van de koeien.
\item
  Geef met twee kleuren aan welke oppervlakken met bovenstaande code
  worden uitgerekend.
\end{itemize}
\EndKnitrBlock{exercise}

Stel, we willen weten welk deel van de populatie koeien minder dan 430
kg weegt. Het gaat dan om het rood gekleurde deel in onderstaande
grafiek:

\includegraphics{{[}03{]}_normal_distribution_files/figure-latex/unnamed-chunk-5-1.pdf}

\BeginKnitrBlock{exercise}
\protect\hypertarget{exr:koe2}{}{\label{exr:koe2} }Koeien

\begin{itemize}
\tightlist
\item
  Bereken de fractie koeien lichter dan 430 kg.
\end{itemize}
\EndKnitrBlock{exercise}

\section{Hypotheses toetsen}\label{hypotheses-toetsen}

Een boer vermoedt dat zijn koeien lichter zijn dan de gemiddelde
Holsteiner-Friesian. Hij zet een koe op de weegschaal, en leest 535 kg
af.

\BeginKnitrBlock{exercise}
\protect\hypertarget{exr:koe3}{}{\label{exr:koe3} }Koe, deel 3

\begin{itemize}
\tightlist
\item
  Bereken m.b.v. pnorm() hoe waarschijnlijk het is dat je een koe hebt
  die 535 kg of minder weegt bij een populatiegemiddelde van 540 kg en
  standaarddeviatie van 5 kg.
\end{itemize}
\EndKnitrBlock{exercise}

Bovenstaande opgave is de essentie van een statistische toets. Je stelt
een nulhypothese op:

\begin{quote}
Het gewicht van de koeien van de boer wijkt niet af van het gewicht van
Holstein-Friesian koeien.
\end{quote}

Blijkt je steekproef heel onwaarschijnlijk te zijn onder de
nulhypothese, dan verwerp je deze, en neem je de alternatieve hypothese
aan. In dit geval gaat het om een eenzijdige toets (de boer denkt dat
zijn koeien lichter zijn).

In het geval dat de boer vermoedt dat het gewicht \textbf{afwijkt} van
Holstein-Friesians, dan hebben we het over een tweezijdige toets. De
kans op een minstens zo grote afwijking als gevonden kan twee kanten op
zijn:

\begin{itemize}
\tightlist
\item
  lichter dan 535 kg
\item
  zwaarder dan 545 kg
\end{itemize}

Hieronder is die kans geïllustreerd:

\includegraphics{{[}03{]}_normal_distribution_files/figure-latex/unnamed-chunk-6-1.pdf}

Je ziet dus dat als je tweezijdig toetst, de waarschijnlijk dat minstens
zo'n grote afwijking voorkomt, twee keer zo groot is dan als je
eenzijdig toetst.

De boer slaat er zijn statistiekaantekeningen op na, en komt tot de
conclusie dat hij een grotere steekproef moet nemen. Hij zet 6 koeien op
een weegbrug en leest 3216 kg af, dus het steekproefgemiddelde is 536
kg.

\BeginKnitrBlock{exercise}
\protect\hypertarget{exr:koe4}{}{\label{exr:koe4} }Koe, deel 4

\begin{itemize}
\tightlist
\item
  De standaarddeviatie voor het gewicht van één Holstein-Friesiankoe is
  5 kg. Wat wordt de standaarddeviatie voor een steekproef van 6 koeien?
\item
  Bereken, met deze standaarddeviatie de waarschijnlijk van maximaal een
  gemiddeld gewicht van 536 als de nulhypothese waar is?
\item
  Bereken de waarschijnlijkheid als je tweezijdig toetst.
\end{itemize}
\EndKnitrBlock{exercise}

\chapter{t-toets voor één steekproef}\label{t-toets-voor-een-steekproef}

\BeginKnitrBlock{ABD}
Lees Paragraph 11.3 (\emph{The one-sample t-test})
\EndKnitrBlock{ABD}

\section{Geschatte standaarddeviatie}\label{geschatte-standaarddeviatie}

In de vorige les hebben jullie met de normale verdeling gerekend.
Hiermee kan je de waarschijnlijkheid berekenen of een bepaalde
steekproef afwijkt van een theoretische waarde.

Allemaal leuk in theorie, maar de praktijk is weerbarstig. Dat is niet
omdat variatie zich niet volgens een normale kansverdeling gedraagt,
maar omdat je de standaarddeviatie schat aan de hand van je steekproef.

Als voorbeeld gebruiken we weer het gewicht van de koeien. Het gewicht
is gemiddeld 540 kg met een standaarddeviatie van 5 kg.

\BeginKnitrBlock{exercise}
\protect\hypertarget{exr:stdev}{}{\label{exr:stdev} }standaardeviatie
schatten

\begin{itemize}
\tightlist
\item
  Voer de volgende functie uit:

  \begin{itemize}
  \tightlist
  \item
    \texttt{s\ \textless{}-\ replicate(10000,\ sd(rnorm(3,\ 540,\ 5)))}
  \end{itemize}
\item
  Nu heb je 10000 keer een steekproef getrokken uit de populatie van de
  koeien, en van iedere steekproef heb je de standaarddeviatie berekend.
  Maak hier nu een histogram van (op de snelle manier):

  \begin{itemize}
  \tightlist
  \item
    \texttt{hist(s)}
  \end{itemize}
\item
  Wat valt je op?
\end{itemize}
\EndKnitrBlock{exercise}

\section{Z-toets}\label{z-toets}

Zoals je in de vorige oefening zag, is gemiddeld je schatting van de
standaarddeviatie kleiner dan de werkelijke standaarddeviatie.

Als je dan, op basis van de geschatte standaarddeviatie, de
kansverdeling van je nulhypothese opzet, dan is die klokvormige
kansverdeling smaller dan in werkelijkheid. Dus verwerp je sneller je
nulhypothese.

Als voorbeeld weer de koeien.

De volgende code voert 10000 keer een Z-toets uit op random steekproeven
van n=3, met de juiste standaarddeviatie:

\begin{Shaded}
\begin{Highlighting}[]
\CommentTok{#Functie om steekproef te nemen en z-toets erop uit te voeren}
\NormalTok{z_sigma <-}\StringTok{ }\NormalTok{function(n, mean, sigma) \{}
  \NormalTok{sample <-}\StringTok{ }\KeywordTok{rnorm}\NormalTok{(n, mean, sigma)}
  \NormalTok{p <-}\StringTok{ }\KeywordTok{pnorm}\NormalTok{(}\KeywordTok{mean}\NormalTok{(sample), mean, sigma/}\KeywordTok{sqrt}\NormalTok{(n))}
\NormalTok{\}}

\CommentTok{#10000 keer bovenstaande functie uitvoeren}
\NormalTok{ps <-}\StringTok{ }\KeywordTok{replicate}\NormalTok{(}\DecValTok{10000}\NormalTok{, }\KeywordTok{z_sigma}\NormalTok{(}\DataTypeTok{n=}\DecValTok{3}\NormalTok{, }\DataTypeTok{mean=}\DecValTok{540}\NormalTok{, }\DataTypeTok{sigma=}\DecValTok{5}\NormalTok{))}

\CommentTok{#welke fractie van de samples verwerpt (ten onrechte!) de H0?}
\NormalTok{alpha=}\FloatTok{0.05}
\KeywordTok{length}\NormalTok{(ps[ps<alpha])/}\KeywordTok{length}\NormalTok{(ps)}
\end{Highlighting}
\end{Shaded}

\begin{verbatim}
## [1] 0.0519
\end{verbatim}

\BeginKnitrBlock{exercise}
\protect\hypertarget{exr:ztoets}{}{\label{exr:ztoets} }Z\_sigma

\begin{itemize}
\tightlist
\item
  Neem bovenstaande code over in een script, en voer uit.
\item
  Varieer met de \(\alpha\) en check of de fractie significant
  afwijkende steekproeven de \alpha volgt.
\end{itemize}
\EndKnitrBlock{exercise}

Stel nu dat we de standaarddeviatie van de populatie niet weten, dan
moeten we die schatten vanuit de steekproef. Wat heeft dat voor
consequenties voor de fout van de eerste soort (ten onrechte H0
verwerpen)?

Dat kan je testen met de volgende code:

\begin{Shaded}
\begin{Highlighting}[]
\CommentTok{#Functie om steekproef te nemen en z-toets erop uit te voeren}
\NormalTok{z_sigma <-}\StringTok{ }\NormalTok{function(n, mean, sigma) \{}
  \NormalTok{sample <-}\StringTok{ }\KeywordTok{rnorm}\NormalTok{(n, mean, sigma)}
  \NormalTok{p <-}\StringTok{ }\KeywordTok{pnorm}\NormalTok{(}\KeywordTok{mean}\NormalTok{(sample), mean, }\KeywordTok{sd}\NormalTok{(sample)/}\KeywordTok{sqrt}\NormalTok{(n))}
\NormalTok{\}}

\CommentTok{#10000 keer bovenstaande functie uitvoeren}
\NormalTok{ps <-}\StringTok{ }\KeywordTok{replicate}\NormalTok{(}\DecValTok{10000}\NormalTok{, }\KeywordTok{z_sigma}\NormalTok{(}\DataTypeTok{n=}\DecValTok{3}\NormalTok{, }\DataTypeTok{mean=}\DecValTok{540}\NormalTok{, }\DataTypeTok{sigma=}\DecValTok{5}\NormalTok{))}

\CommentTok{#welke fractie van de samples verwerpt (ten onrechte!) de H0?}
\NormalTok{alpha=}\FloatTok{0.05}
\KeywordTok{length}\NormalTok{(ps[ps<alpha])/}\KeywordTok{length}\NormalTok{(ps)}
\end{Highlighting}
\end{Shaded}

\begin{verbatim}
## [1] 0.1208
\end{verbatim}

\BeginKnitrBlock{exercise}
\protect\hypertarget{exr:zstschat}{}{\label{exr:zstschat} }Z\_s

\begin{itemize}
\tightlist
\item
  Neem bovenstaande code over in een script, en voer uit.
\item
  Varieer met de \(\alpha\) en check of de fractie significant
  afwijkende steekproeven de \(\alpha\) volgt.
\item
  Herhaal met n=30 (dus een steekproefgrootte van 30)
\item
  Wat is het verschil?
\end{itemize}
\EndKnitrBlock{exercise}

\section{t-verdeling}\label{t-verdeling}

In de vorige oefening zag je het effect van een te kleine schatting van
je standaarddeviatie op het voorkomen van een type-1-fout (ten onrechte
verwerpen van de Ho). Dit is vooral een probleem bij kleine
steekproeven. In een beroemd wetenschappelijk artikel van een
mederwerker van Guinness (onder pseudoniem \emph{Student}), wordt dit
probleem uitgelegd en een oplossing gegeven. De oplossing is om een
aangepaste kansverdeling te gebruiken: de \textbf{t-verdeling}:

\includegraphics{{[}04{]}_one_sample_T-test_files/figure-latex/unnamed-chunk-4-1.pdf}

In rood is de t-verdeling en met een zwarte lijn is de normale verdeling
geplot. Je ziet dat de staarten van de t-verdeling dikker zijn dan de
normale verdeling. In de volgende paragraaf gaan we aan de slag met de
t-toets die i.t.t. de z-toets wel rekening houdt met het effect van
steekproefgrootte op de kansverdeling.

\section{one-sample t-test in R}\label{one-sample-t-test-in-r}

De functie voor de t-toets in R is \texttt{t.test()}. Deze functie heeft
als \emph{argument} een vector (rij data) nodig. Daarnaast kan je onder
andere de volgende opties aangeven:

\begin{itemize}
\tightlist
\item
  mu = \ldots{} (een getal waartegen je de dataset test, standaard is 0)
\item
  alternative = (kies uit ``greater'' of ``less'' als je eenzijdig wilt
  testen)
\end{itemize}

Als voorbeeld willen we testen of het gemiddelde voor biocalculus1 hoger
ligt dan 6:

\begin{Shaded}
\begin{Highlighting}[]
\KeywordTok{library}\NormalTok{(readxl)}
\NormalTok{cijfers <-}\StringTok{ }\KeywordTok{read_excel}\NormalTok{(}\StringTok{"../data/cijfers.xlsx"}\NormalTok{)}
\end{Highlighting}
\end{Shaded}

\begin{Shaded}
\begin{Highlighting}[]
\KeywordTok{t.test}\NormalTok{(cijfers$biocalculus, }\DataTypeTok{mu=}\DecValTok{6}\NormalTok{, }\DataTypeTok{alternative =} \StringTok{"greater"}\NormalTok{)}
\end{Highlighting}
\end{Shaded}

\begin{verbatim}
## 
##  One Sample t-test
## 
## data:  cijfers$biocalculus
## t = 3.3258, df = 205, p-value = 0.0005223
## alternative hypothesis: true mean is greater than 6
## 95 percent confidence interval:
##  6.215195      Inf
## sample estimates:
## mean of x 
##   6.42767
\end{verbatim}

In bovenstaande output zie je de berekende t-waarde (\emph{t}), het
aantal vrijheidsgraden (\emph{df}) en de p-waarde (\emph{p-value}).
Verder ook het betrouwbaarheidsinterval (die voor een eenzijdige toets
naar oneindig loopt (\emph{inf}) en het gemiddelde van de steekproef.

Om te checken of de t-toets wel een betrouwbare p-waarde geeft voor de
waarschijnlijkheid dat de H0 waar is, kan je de volgende code uitvoeren:

\begin{Shaded}
\begin{Highlighting}[]
\CommentTok{#Nu met t-toets}
\NormalTok{ttoets <-}\StringTok{ }\NormalTok{function(n, mu, sigma) \{}
  \NormalTok{sample <-}\StringTok{ }\KeywordTok{rnorm}\NormalTok{(n, mu, sigma)}
  \KeywordTok{return}\NormalTok{(}\KeywordTok{t.test}\NormalTok{(sample, }\DataTypeTok{mu=}\NormalTok{mu)$p.value)}
\NormalTok{\}}

\CommentTok{#10000 keer bovenstaande functie uitvoeren}
\NormalTok{tp <-}\StringTok{ }\KeywordTok{replicate}\NormalTok{(}\DecValTok{10000}\NormalTok{, }\KeywordTok{ttoets}\NormalTok{(}\DataTypeTok{n=}\DecValTok{3}\NormalTok{, }\DataTypeTok{mu=}\DecValTok{540}\NormalTok{, }\DataTypeTok{sigma=}\DecValTok{5}\NormalTok{)) }

\CommentTok{#Fractie gevallen waarin H0 ten onrechte wordt verworpen}
\NormalTok{alpha <-}\StringTok{ }\FloatTok{0.05}
\KeywordTok{length}\NormalTok{(tp[tp<alpha])/}\KeywordTok{length}\NormalTok{(tp)}
\end{Highlighting}
\end{Shaded}

\begin{verbatim}
## [1] 0.0492
\end{verbatim}

In 1996 heeft statistiekdocent Allen Shoemaker een dataset met
lichaamstemperatuur, hartslag en geslacht van 130 studenten
gepubliceerd. Op blackboard vind je een naar Celsius omgerekende versie
hiervan.

\BeginKnitrBlock{exercise}
\protect\hypertarget{exr:onesamplettest}{}{\label{exr:onesamplettest}
}Eenzijdige t-toets

\begin{itemize}
\tightlist
\item
  Test met een eenzijdige t-toets of de gemiddelde lichaamstemperatuur
  afwijkt van 37°.

  \begin{itemize}
  \tightlist
  \item
    Schrijf eerste de nulhypothese en alternatieve hypothese op.
  \item
    Voer de toets en geef de conclusie.
  \end{itemize}
\end{itemize}
\EndKnitrBlock{exercise}

\chapter{Gepaarde t-toets}\label{gepaarde-t-toets}

\BeginKnitrBlock{ABD}
Lees Paragraph 12.1 (\emph{Paired sample versus two independent
samples})

Lees Paragraph 12.2 (\emph{Paired comparison of means})
\EndKnitrBlock{ABD}

Met een enkelvoudige t-toets (\emph{one-sample t-test}) bereken je de
overschrijdingskans van een steekproef t.o.v. een \textbf{normaal}
verdeelde waarde. Een voorbeeld van vorige week: ``wat is de kans op een
gemiddeld gewicht van een steekproef van 5 koeien van kleiner of gelijk
aan 530 kg, aangenomen dat een koe gemiddeld 540 kg weegt?''.

In de vorige les hebben jullie ontdekt dat de kansverdeling van een
steekproefgemiddelde alleen een normale verdeling volgt als je van
tevoren weet wat de standaarddeviatie is in je populatie. Wanneer je dat
niet weet, en de standaarddeviatie schat aan de hand van je steekproef,
dan wijkt de verdeling iets af van de normale verdeling. Deze
kansverdeling volgt een t-verdeling. Door een t-toets te gebruiken,
i.p.v. een z-toets, kan je wel de juiste overschrijdingskans berekenen.

\section{Gepaarde data}\label{gepaarde-data}

De \emph{one-sample t-test} kan je ook gebruiken als je verschillen
tussen twee behandelingen wil laten zien als de data \textbf{gepaard}
is. Je laat bevoorbeeld studenten een wiskundetoets doen voor en na een
paar glazen bier. Natuurlijk verwacht je dat je rekencapaciteit
achteruit gaat na het drinken van bier (toch?), De hypotheses zijn:

\begin{quote}
H\textsubscript{0}: Score vóór drinken is gelijk aan score ná het
drinken van bier.
\end{quote}

\begin{quote}
H\textsubscript{1}: Score vóór drinken is lager dan score ná het drinken
van bier.
\end{quote}

Onder de H\textsubscript{0} verwacht je dat de score voor en na het
drinken van bier gemiddeld gelijk is. Met andere woorden: het verschil
tussen voor en na is gemiddeld 0. Stap 1 is voor iedere student het
verschil te berekenen tussen beide testen. Stap 2 is om te berekenen wat
de overschrijdingskans is van deze uitkomst als de H\textsubscript{0}
waar is. Dat doe je met een \emph{one-sample t-test}.

Maar deze twee stappen kan je ook in één keer uitvoeren met een gepaarde
t-toets (\emph{paired t-test}).

\section{Gepaarde t-toets}\label{gepaarde-t-toets-1}

Belangrijk met gepaarde data is dat het helder is welke metingen bij
elkaar horen. Dat doe je door op de volgende manier je data te ordenen:

\begin{tabular}{r|r|r}
\hline
student & voor & na\\
\hline
1 & 7 & 5\\
\hline
2 & 4 & 4\\
\hline
3 & 6 & 5\\
\hline
4 & 9 & 7\\
\hline
5 & 5 & 5\\
\hline
\end{tabular}

Wanneer je de data ingelezen hebt in R (bijv. als objectnaam
\texttt{bier}), kan je met de volgende functie een gepaarde t-toets
uitvoeren:

\begin{Shaded}
\begin{Highlighting}[]
\KeywordTok{t.test}\NormalTok{(bier$voor, bier$na, }\DataTypeTok{paired =} \OtherTok{TRUE}\NormalTok{)}
\end{Highlighting}
\end{Shaded}

\BeginKnitrBlock{exercise}
\protect\hypertarget{exr:machomerels}{}{\label{exr:machomerels} }Macho
merels

\begin{itemize}
\tightlist
\item
  Download met de volgende functie de data van de \emph{Red-winged
  Blackbirds}:

  \begin{itemize}
  \tightlist
  \item
    \texttt{blackbird\ \textless{}-\ read.csv(url("http://www.zoology.ubc.ca/\textasciitilde{}schluter/WhitlockSchluter/wp-content/data/chapter12/chap12e2BlackbirdTestosterone.csv"))}
  \end{itemize}
\item
  Voor dezelfde toets uit als in het boek. Check evt. de opties voor de
  functies via de code \texttt{?t.test}
\end{itemize}
\EndKnitrBlock{exercise}

\BeginKnitrBlock{exercise}
\protect\hypertarget{exr:oef_H11}{}{(\#exr:oef\_H11) }Maak de volgende
oefeningen uit het boek:

\begin{itemize}
\tightlist
\item
  11.5
\item
  11.11
\end{itemize}
\EndKnitrBlock{exercise}

\chapter{Onafhankelijke t-toets:}\label{onafhankelijke-t-toets}

\BeginKnitrBlock{ABD}
Lees Paragraph 12.3 (\emph{Two-sample comparison of means}) NB: je hoeft
niet de berekeningen te kennen.
\EndKnitrBlock{ABD}

Bij een \emph{one-sample t-test} en een gepaarde t-toets heb je te maken
met één rij data waarvan je wilt weten of het gemiddelde ervan
significant afwijkt van een bepaalde waarde. Maar vaak heb je te maken
met meerdere groepen waarvan je wilt weten of ze écht van elkaar
verschillen. De nulhypothese is dan dat het gemiddelde van de groepen
niet van elkaar verschillen. Dat is een subtiel verschil met de gepaarde
t-toets waarbij je voor iedere herhaling verwacht dat eerste en tweede
behandeling gelijk aan elkaar zijn.

\BeginKnitrBlock{exercise}
\protect\hypertarget{exr:welnietgepaard}{}{\label{exr:welnietgepaard}
}Gepaard of niet?

\begin{itemize}
\tightlist
\item
  Los \emph{Practice problem} 12.4 (blz. 355) op
\end{itemize}
\EndKnitrBlock{exercise}

\section{Standaarddeviatie}\label{standaarddeviatie}

Bij een \emph{one-sample t-test} is het recht-toe-recht-aan om de
standaarddeviatie te schatten. Je hebt één rij data waarvan je
gemiddelde en steekproefstandaarddeviatie berekent. Aan de hand daarvan
bepaal je hoe waarschijnlijk je afwijking is ten opzichte van je
nulhypothese.

In de traditionele \textbf{onafhankelijke t-toets} (\emph{independent
t-test}) wordt de standaarddeviatie van het verschil in de gemiddelde
waardes van beide groepen geschat via een gewogen gemiddelde van de
standaarddeviaties in beide groepen. Dat werkt alleen als de
standaarddeviatie niet te veel verschilt tussen de twee groepen.

Verschilt de standaarddeviatie flink van elkaar, dan kan je bovenstaande
aanpak niet gebruiken. Je moet dan een alternatieve toets gebruiken, de
\textbf{Welch's t-toets}. Deze toets maakt niet de aanname dat de
standaarddeviatie gelijk is in beide groepen.

In de praktijk wordt tegenwoordig aangeraden om \textbf{altijd} de
Welch's t-toets te gebruiken, tenzij je sterke aanwijzing hebt dat je de
standaardeviaties voor beide groepen gelijk zijn. Zie ook het volgende
\href{https://academic.oup.com/beheco/article/17/4/688/215960}{artikel}.

\section{T-toets in R}\label{t-toets-in-r}

De \emph{one-sample t-test}, de gepaarde t-toets, de onafhankelijke
t-toets en de Welch's t-toets zijn allemaal uit te voeren met de
volgende functie: \texttt{t.test()}.

In dit hoofdstuk willen we twee groepen vergelijken. Dat kan je op twee
manieren aangeven in \texttt{t.test()}:

\begin{itemize}
\tightlist
\item
  \texttt{t.test(g1,\ g2)}
\item
  \texttt{t.test(v\textasciitilde{}g)}
\end{itemize}

De eerste optie is hetzelfde als je bij de gepaarde t-toets hebt gezien:
g1 en g2 zijn twee vectoren. In tegenstelling tot bij de gepaarde
t-toets, hoeft bij een onafhankelijke t-toets of Welch's t-toets beide
vectoren niet even lang te zijn.

De tweede optie heeft de voorkeur. Waarom? Omdat die aansluit bij de
standaardmanier om je data te organiseren. Heb je bijvoorbeeld de lengte
van planten gemeten in twee groepen (bijv, met of zonder bemesting), dan
is dit de manier om je data in een tabel te zetten:

\begin{tabular}{l|r}
\hline
behandeling & plantlengte\\
\hline
controle & 15\\
\hline
controle & 17\\
\hline
controle & 16\\
\hline
controle & 14\\
\hline
controle & 16\\
\hline
bemest & 15\\
\hline
bemest & 23\\
\hline
bemest & 21\\
\hline
bemest & 27\\
\hline
bemest & 18\\
\hline
\end{tabular}

Met de tweede optie (\texttt{v\textasciitilde{}g}) geef je aan dat de
vector verdeeld is in groepen (aangegeven met de variabele g). In het
geval van het bemestingsexperiment voor je met de volgende code de
Welch' t-toets uit:
\texttt{t.test(plantdata\$plantlengte\textasciitilde{}plantdata\$behandeling)}.
De standaardinstelling van R voor \texttt{t.test()} is dat er
\textbf{geen} gelijke variantie wordt aangenomen, dus de Welch't-toets
wordt uitgevoerd. Wil je wel een `gewone' onafhankelijke t-toets
uitvoeren, moet je expliciet aangeven dat er gelijke standaarddeviatie
wordt aangenomen voor beide groepen (`t.test(v\textasciitilde{}g,
var.equal = TRUE)).

\BeginKnitrBlock{exercise}
\protect\hypertarget{exr:bemesting}{}{\label{exr:bemesting} }Bemesting

\begin{itemize}
\tightlist
\item
  Voer de data uit bovenstaande tabel in een Excel file
\item
  Formuleer de H\textsubscript{0} en H\textsubscript{1}
\item
  Maak een script die de data importeert en een t-toets uitvoert.
\item
  Wat is je conclusie?
\end{itemize}
\EndKnitrBlock{exercise}

\BeginKnitrBlock{exercise}
\protect\hypertarget{exr:muggen}{}{\label{exr:muggen} }Bier en muggen

\begin{itemize}
\tightlist
\item
  Voer \emph{Practise problem} 12.16 (blz. 359) uit
\end{itemize}
\EndKnitrBlock{exercise}

\BeginKnitrBlock{exercise}
\protect\hypertarget{exr:oef_H12}{}{(\#exr:oef\_H12) }Maak de volgende
oefeningen uit het boek:

\begin{itemize}
\tightlist
\item
  12.4
\item
  12.5
\item
  12.8
\item
  12.11
\end{itemize}
\EndKnitrBlock{exercise}


\end{document}
